% Created 2022-01-11 Tue 14:27
% Intended LaTeX compiler: 
\documentclass[a4paper]{article}
\usepackage[utf8]{inputenc}
\usepackage[T1]{fontenc}
\usepackage{graphicx}
\usepackage{grffile}
\usepackage{longtable}
\usepackage{wrapfig}
\usepackage{rotating}
\usepackage[normalem]{ulem}
\usepackage{amsmath}
\usepackage{textcomp}
\usepackage{amssymb}
\usepackage{capt-of}
\usepackage{hyperref}
\usepackage{xcolor}

\usepackage{rmk_org_cv}

\author{Rainer M. Krug}
\date{\today}
\title{CV Rainer M. Krug}
\hypersetup{
 pdfauthor={Rainer M. Krug},
 pdftitle={CV Rainer M. Krug},
 pdfkeywords={},
 pdfsubject={},
 pdfcreator={}, 
 pdflang={English}}
\begin{document}

\maketitle


\section{Personal Details}

\subsection{Name}

Rainer M. Krug

\subsection{Address}

Soorhaldenstr. 12 \\
8308 Illnau \\
Switzerland

\subsection{ORCID}

\href{https://orcid.org/0000-0002-7490-0066}{0000-0002-7490-0066}

\subsection{Mobile}

+41 78 630 66 57

\subsection{E-mail}

Rainer.Krug@ieu.uzh.ch (work)

Rainer@krugs.de (private)

\subsection{Date of birth}

12 April 1968

\subsection{Civil status}

married, one daughter

\section{Languages}

\textbf{German} native language

\textbf{English} reading, writing and speaking fluent

\textbf{French} reading, writing and speaking fair


\section{Qualifications}

\subsection{Postgraduate}


\subsection{2008}

\textbf{PhD in Conservation Ecology}, Stellenbosch University, South Africa

Thesis Title: Modelling seed dispersal in restoration and biological
invasion.

\subsection{1997}

\textbf{MSc Conservation Biology}, University of Cape Town, South Africa

Thesis topic: The Genetic Diversity in a Founded Population of the
African buffalo (\emph{Syncerus caffer}): an example of an Artificial
Bottleneck.

\subsection{1995}

\textbf{Diplom (MSc equivalent) in Physics}, Philips-Universität Marburg, Germany

Thesis Title: Der Einfluss von Habitat Heterogenität auf die mittlere
Überlebensdauer von Populationen (The influence of habitat heterogeneity
on the mean survival time of populations)\\
Subjects for oral examination: Experimental Physics, Theoretical
Physics, Ecological Modelling, Biology

\subsection{Undergraduate}

\subsection{1992}

\textbf{Vor-Diplom (BSc equivalent) in physics}, Philips-Universität Marburg, Germany

Subjects for oral examination: Experimental Physics, Theoretical Physics, Mathematics, Chemistry.

\section{Positions held}

\subsection{03/2017--present}

\textbf{Researcher / Research Assistant}

Department of Evolutionary Biology and Environmental Studies, University Zürich

\subsection{08/2015--09/2015}

\textbf{Postdoctoral Researcher} 

Laboratoire Ecologie, Systematique et Evolution, Paris Sud

\subsection{11/2014--12/2014}

\textbf{Postdoctoral Researcher} 

Laboratoire Ecologie, Systematique et Evolution, Paris Sud

\subsection{09/2013--11/2013}

\textbf{Postdoctoral Researcher}

Laboratoire Ecologie, Systematique et Evolution, Paris Sud

\subsection{08/2011--12/2016}

\textbf{Research Associate}

DST-NRF Centre of Excellence for Invasion Biology, Stellenbosch University

\subsection{2009--2015 }

\textbf{Child care ranging from 50\%--100\%}


\subsection{06/2008--06/2008}

\textbf{Postdoctoral Research Fellow, hosted by Prof.\ Dave Richardson.}

DST-NRF Centre of Excellence for Invasion Biology, Stellenbosch University

\subsection{06/2007--06/2008}

\textbf{Postdoctoral Research Fellow, hosted by Prof.\ Timm Hoffman.} 

Plant Conservation Unit, University of Cape Town

\subsection{1999}
\textbf{Accident induced time off}
One year in Germany due to hospital visits and cancelation of started and financed PhD.


\section{Other Positions / Roles}

\subsection{2022}
Invited Academic Guest Editor for paper in the field of reporoducible methods


\subsection{2021--present}
Fellow of the Center for Reproducible Science (CRS) at UZH

\subsection{2021--present}

Co-Chair of the \href{https://intgovforum.org/en/content/dynamic-coalition-on-environment-dce}{Dynamic Coalition on Environment (DCE)} of the \href{https://intgovforum.org/en}{Internet Governance Forum (IGF)}

\subsection{2021}

Member of \href{https://www.intgovforum.org/en/content/policy-network-on-environment-pne}{Policy Network on Environment (PNE)}, Intersessional workstream of the IGF

\subsection{2019--present}

Elected Expert Member of the \href{https://ipbes.net/}{Intergovernmental Science-Policy Platform on Biodiversity and Ecosystem Services (IPBES)} \href{https://ipbes.net/knowledge-data}{Task Force on Knowledge and Data}, nominated by Switzerland

\subsection{2017--2019}

IPBES Chapter Scientist of the chapter 4 of the IPBES Global Assessment

\subsection{08/2011--12/2016}

Research Associate DST-NRF Centre of Excellence for Invasion Biology, Stellenbosch University


\section{Research Projects}

\begin{itemize}[leftmargin=1.1in]

\item Development of an R package for running a simulation in a
	transparent and reproducible way, including the running of the
	analysis, saving and reloading of parameter and results as well as
	support for analysis and graphing of the results using RMarkdown. (IEU UZH)

\item Development, implementation and maintenance of an automatic data
	pre-processing and extraction pipeline to extract Research Ready
	(data which will be used in forecasting in this case) data from
	regularly measured data and movies, recorded in open (but
	difficult to use) and proprietary formats. The data is converted in
	open formates and saved as Archive Ready (i.e. FAIR) data and further
	processed into Research Ready data which, in the final step,  is
	further analysed by other researchers in the team. All elements of
	the pipeline are implemented in a completely transparent and
	reproducible way, written in R (several R packages to
	compartmentalise the pipeline), which is running in a docker
	container and controlled using makefiles and bash scripts. The docker
	container runs on a S3IT Science Cloud instance which is also
	accessible via a Samba server. (IEU UZH)

\item Development of an approach to develop domain specific metadata
	schemes, implemented in an accompanying R packege (dmdScheme). This
	approach was used to develop a domain specific metadata scheme for
	the experiments (aquatic microbial microcosm experiments), which is
	implemented in the package emeScheme and accompanying metadata scheme
	definition. The resulting emeScheme is used for the ongoing
	experiments in our group as well as retrospectively to make the data
	availabe on public repositories. (IEU UZH)

\item Development of a data management strategy for our research group
	and of tools to facilitate the provision and improve the quality of
	metadata. (IEU UZH)

%% \item Predictions in Chaotic systems

\item Management of the large scale Literature Analysis for
	(\textbf{IPPBES Global Assessment Chapter 4}) (more than 20 authors,
	> 5000 initial papers and a complex questionaire in excel sheet to
	collect the results from the review) and the generated data and from
	the Excel Database in a reproducible way for archiving as well as
	graphing of the results. (Laboratoire Ecologie, Systematique et Evolution, Paris
	Sud and IEU UZH)

\item Analysis of measured vertical wind profiles to improve the
	performance of a forest growth model (CASTANEA) in regards to energy
	balance (Laboratoire Ecologie, Systematique et Evolution, Paris
	Sud)

\item Modelling temporal and spatial dynamics of a range of different
	different alien species, alien control agents and management
	strategies (Centre for Invasion Biology, Stellenbosch
	University)

\item Optimising alien invasive plant management through modelling of
	temporal and spatial modelling (Centre for Invasion Biology,
	Stellenbosch University)

\item Modelling the role of seed dispersal in restoration and
	biological invasion (Department of Conservation Ecology and
	Entomology, Stellenbosh University)

\end{itemize}

\section{Reproducibility and Data Management}
\begin{itemize}[leftmargin=1.1in]

\item As \textbf{elected expert of the IPBES Knowledge and Data Task
	Force} I led the development of the Data and Knowledge Management
	Policy of IPBES. This is based based on FAIR and CARE data
	managememnt principles, which requires IPBES to follow reproducible
	workflows in all research related activities. The policy further
	requires IPBES to provide support to implement the policy to its
	experts which is mainly done by the Data TSU (Technical Support
	Unit). In addition, I am also working with the tsu on the develop of
	teaching and information material as well as technical guidelines to
	implement at all stages of the research process reproducible and open
	workflows. One direction the Task Force and the TSU are working
	towards is to use dynamic reports (on demand as well as in real-time
	updated) in IPBES. Additionally, we developed the Data Vision 2030, 
	defining a vision of how IPBES data should be managed and published, 
	including milestones and tasks. 

\item Between 2015 and 2017 I was a \textbf{chapter scientist} for the
	Global Assessment, Chapter 4. Here, I managed the large scale
	literature review for the IPBES Global Assessment Chapter 4,
	including data collection, quality control, analysis, graphing and
	archival (finalising).

\item In 2021 I was active in the  in the \textbf{Policy Network on
	Environment} (PNE) (Internet Governance Forum) where I successfully
	promoted the FAIR use of data (Findable, Accessible, Interoperable
	and Reusable) as well as the CARE principles (Collective benefits,
	Authority of control, Responsibility and Ethics) to be included in
	the recommendations. Caused by the overall positive acceptance of the
	report, it was decided to continue the work in a new Dynamic
	Coalition, the DCE, of which I became a co-chair.

\item After 2022, as the \textbf{co-chair of the DCE}, I am
	promoting the closer integration of environmental, particularly
	biodiversity, considerations and internet governance related
	questions. Combining this aim with my experience in the IPBES
	Knowledge and Data Task Force, particularly the role of FAIR and CARE
	data management, the charta of the DCE includes these open science
	principles as a cornerstone.

\end{itemize}


\section{Teaching Experience}

\begin{itemize}[leftmargin=1.1in]

\item \textbf{Computing consultant} in BIO144 (Data Analysis in
	Biology) at UZH (2021--2023)

\item Development of \textbf{video tutorials} within IPBES regarding
	the data management policy and its implementation for
	interdisciplinary experts from diverse scientific backgrounds

\item Development of \textbf{technical guidelines} for the
	implementation of the data management policy in IPBES for
	interdisciplinary experts from diverse scientific backgrounds

\item teaching a week long \textbf{block course} Introduction to True
	Basic at the University of Cape Town for five years. Students were
	from different scientific backgrounds and from diverse universities
	attending Conservation Biology Masters course.


\end{itemize}

\section{Areas of Interest and expertise}

\subsection{Keywords}

    open science; FAIR; CARE; reproducibility; 
    open source tools; 
    data and metadata management and policy;
     
    combined modelling and experimental approaches;
    ecological modelling; statistical computing; 
	  invasive species management; biodiversity research;

\subsection{Research}


My current focus of work at UZH is in the field of data management,
	metadata management, reproducible workflows and data archival,
	embedded in the principles of open science, FAIR and reproducibility.
	This includes the development of a framework for continuous data
	analysis and preparation for archival which is used by an ongoing
	project, development of data management guidelines for our research
	group, development of a domain specific metadata scheme to make
	entering and useful metadata as easy as possible for the researcher
	while at the same providing all the information needed for a useful
	re-use of the archived data;

    

%% My previous research interest and focus is on 1) spatial modelling and
%%	analysis of pattern and processes and their integration with field
%%	experiments and observations, ranging from population (local) to
%%	ecosystem (regional) scale, 2) the impact of change (climate change,
%%	human impacts, alien spread, \ldots{}) and conscious human actions
%%	(management) on these pattern and ultimately on the function of these
%%	ecosystems and ecosystem services, and 3) the use of models in decision
%%	support of the management of natural resources.

Throughout my research career I have developed and used different types of
	models, ranging from individual based models, hybrid models using
	individual based approaches together with grid based elements, to pure
	grid based models. The systems studied ranged from populations and
	communities on the local scale to community dynamics (e.g.\ 
	grassland-shrubland dynamics, two biocontrol species one invasive species system,
	spread of three alien invasive species) on the local scale and spread
	simulations of individual species on the national scale under different
	climate change or management actions.

All the programming, simulation and analysis tools I use are Open
	Source software (R, GRASS GIS, QGIS). This e.g.\ provides provides me
	with the flexibility to develop the simulation models and analysis
	protocols and pipelines without additional costs, distribute them
	freely and to enable others (scientists as well as other implementing
	agencies and users) to use, evaluate, and develop the code further
	without limitations and without having to purchased specific
	software. 
	
Rmarkdown (lately quarto) is my tool of choice to document and run data
	preparation (e.g.\ conversion and filtering of outliers / fault
	results) and analysis / graphing of data. This `Docs as Code' type of
	document enables one to focus on the documentation and what one wants
	to do and embeds the code as well as the results in the final document.

I use the R package structure as the template for projects. This takes
	the reproducibility even one step further, as dependencies can be
	communicated clearly (and are enforced during installation) and
	thinking in `functions' instead of long scripts is encouraged.
	Consequently I maintain number of R packages for internal as well as
	public usage (see e.g.\ my
	\href{https://rkrug.r-universe.dev/builds}{R-Universe repository}).

At UZH / IEU, my work involves research support,
	ranging from informal discussion on how workflows and scripts could
	be improved in regards to reproducibility as well as execution speed
	and data management aspects, designing and writing R packages to
	simplify or automate workflows up to designing workflows which use
	different computing platforms to the actual management of research
	data to finally submitting it to Zenodo for long term preservation.


\subsection{Open Science}

I am particularly interested in the implementation of open science and
	reproducibility principles, how these can be implemented and how
	perceived hurdles can be overcome in individual groups (e.g. IPBES or
	the research group I am in). 
Apart from the technological aspects of open science and reproducibility, 
  I am interested in the policy level in how these principles can be codified
  in Data Management Policy form in different organisations. The development
  of the IPBES Data and Knowledge Management Policy provided fascinating insight
  into the problems facing non-traditional science institutes in handling data 
  and codify open science and reproducibility principles for a wide variety of different data, ranging from
  quantitative to qualitative data, western science paradigme based to
  indigenous and local knowledge and peer revieweed to gray literature.

\subsection{Software}

In my carrer, I obtained expertise in

\begin{itemize}[leftmargin=1.1in]

\item different programming languages (recently mainly R and bash,
	earlier Delphi / Pascal and rudimentary work in Python and C),

\item linking these through e.g.\ scripting into automated processing
	and data work flows,

\item setting up and usage of virtual computing platforms (most
	recently S3IT Science Cloud) using docker, and virtual machines,

\item using version control, mainly git via command line and through github,

\item setting up and maintenance of small MySQL and sqlite databases as
	well as SAMBA servers and trouble shooting of a wide variety of
	software problems associated with the provision of these programs and
	their maintenance. All setup steps are documented in documents to
	make the re-creation of the instances / servers possible not only for
	me but for others as well.

\item all three major operating systems (Mac and Linux expert
	user, Windows advanced user) and a wide range of programs,
	dominantely but not exclusively open source (Rstudio, Emacs, MS
	Office, Apple office products, GRASS, etc).

\end{itemize}
    

    

    
% \section{Additional skills}

% \subsection{Computer}

% \textbf{Operating System} Expert Mac and Linux user; advanced Windows user

% \textbf{Programming Languages} Extensive experience in programming in R,
% 	Delphi / Pascal; basic experimence in \LaTeX{}; rudimentary experience 
%     in Python and C

% \textbf{Programs} Extensive experience in R (RStudio and Emacs), GRASS, bash; 
%     Daily RStudio and BBEdit user;
%     Zotero;
% 	Apple office programs; MS Office programs; basic experience of QGIS and
% 	Arc-GIS



\section{Grants}

\subsection{2009--2010}

NRF Freestanding Postdoctoral Fellowship

\subsection{1999--2000}

Deutscher Akademischer Austauschdienst (DAAD: German Academic Exchange
	Service) grant to conduct fieldwork for PhD at Gobabeb, Namibia.

\subsection{1996--1997}

Deutscher Akademischer Austauschdienst (DAAD: German Academic Exchange
	Service) grant to attend MSc in Conservation Biology course at UCT.
	
\section{Publications}

Data publications are not included. For an autogenerated list of the all
	publications see \url{https://orcid.org/0000-0002-7490-0066}

\titlespacing{\subsection}
            {0.4\textwidth}% max width of the title(for wrap/leftmargin shape)
            {5pt}% vertical space before the title
            {15pt}% separation between title and text
            
            
\subsection{Articles \& Pre-Prints}

\begin{btSect}[elsarticle-harv]{MP_PeerReviewed}
\btPrintAll
\end{btSect}


\subsection{Book Chapters}

\begin{btSect}[elsarticle-harv]{MP_InBook}
\btPrintAll
\end{btSect}


\subsection{Conference proceedings}

\begin{btSect}[elsarticle-harv]{MP_Proceedings}
\btPrintAll
\end{btSect}


\subsection{Conference presentations}

Only first author, except invited keynote presentations
\begin{btSect}[elsarticle-harv]{MP_Presentations}
\btPrintAll
\end{btSect}

\subsection{Software Packages}

\begin{btSect}[elsarticle-harv]{MP_Software}
\btPrintAll
\end{btSect}


\subsection{Guest lectures}

\begin{btSect}[elsarticle-harv]{MP_GuestLecture}
\btPrintAll
\end{btSect}
\end{document}