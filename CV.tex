% Created 2022-01-11 Tue 14:27
% Intended LaTeX compiler: 
\documentclass[a4paper]{article}
\usepackage[utf8]{inputenc}
\usepackage[T1]{fontenc}
\usepackage{graphicx}
\usepackage{grffile}
\usepackage{longtable}
\usepackage{wrapfig}
\usepackage{rotating}
\usepackage[normalem]{ulem}
\usepackage{amsmath}
\usepackage{textcomp}
\usepackage{amssymb}
\usepackage{capt-of}
\usepackage{hyperref}

\usepackage{rmk_org_cv}
\author{Rainer M. Krug}
\date{\today}
\title{CV Rainer M. Krug}
\hypersetup{
 pdfauthor={Rainer M. Krug},
 pdftitle={CV Rainer M. Krug},
 pdfkeywords={},
 pdfsubject={},
 pdfcreator={}, 
 pdflang={English}}
\begin{document}

\maketitle


\section{Personal Details}

\subsection{Name}

Rainer M. Krug
\subsection{Address}

Soorhaldenstr. 12 \\
8308 Illnau \\
Switzerland

\subsection{ORCID}
\href{https://orcid.org/0000-0002-7490-0066}{0000-0002-7490-0066}

\subsection{Telephone}

+41 52 534 65 13
\subsection{Mobile}

+41 78 630 66 57
\subsection{E-mail}

Rainer@krugs.de
Rainer.Krug@ieu.uzh.ch
\subsection{Date of birth}

12 April 1968
\subsection{Civil status}

married, one daughter

\section{Qualifications}

\subsection{Postgraduate}


\subsection{2008}

\textbf{PhD in Conservation Ecology}, Stellenbosch University, South Africa

Thesis Title: Modelling seed dispersal in restoration and biological
invasion.

\subsection{1997}

\textbf{MSc Conservation Biology}, University of Cape Town, South Africa

Thesis topic: The Genetic Diversity in a Founded Population of the
African buffalo (\emph{Syncerus caffer}): an example of an Artificial
Bottleneck.

\subsection{1995}

\textbf{Diplom (MSc equivalent) in Physics}, Philips-Universität Marburg, Germany

Thesis Title: Der Einfluss von Habitat Heterogenität auf die mittlere
Überlebensdauer von Populationen (The influence of habitat heterogeneity
on the mean survival time of populations)\\
Subjects for oral examination: Experimental Physics, Theoretical
Physics, Ecological Modelling, Biology

\subsection{Undergraduate}



\subsection{1992}

\textbf{Vor-Diplom (BSc equivalent) in physics}, Philips-Universität Marburg, Germany

Subjects for oral examination: Experimental Physics, Theoretical Physics, Mathematics, Chemistry.

\section{Positions held}

\subsection{03/2017 -- present}

\textbf{Department of Evolutionary Biology and Environmental Studies, University Zürich}

Researcher

\subsection{08/2015 -- 09/2015}

\textbf{Laboratoire Ecologie, Systematique et Evolution, Paris Sud} 

Postdoctoral Researcher

\subsection{11/2014 -- 12/2014}

\textbf{Laboratoire Ecologie, Systematique et Evolution, Paris Sud} 

Postdoctoral Researcher

\subsection{09/2013 -- 11/2013}

\textbf{Laboratoire Ecologie, Systematique et Evolution, Paris Sud}

Postdoctoral Researcher

\subsection{08/2011 -- 12/2016}

\textbf{DST-NRF Centre of Excellence for Invasion Biology, Stellenbosch University}

Research Associate

\subsection{06/2008 -- 06/2008}

\textbf{DST-NRF Centre of Excellence for Invasion Biology, Stellenbosch University}

Postdoctoral Research Fellow, hosted by Prof. Dave Richardson.

\subsection{06/2007 -- 06/2008}

\textbf{Plant Conservation Unit, University of Cape Town} 

Postdoctoral Research Fellow, hosted by Prof. Timm Hoffman.

\section{Other Positions / Roles}

\subsection{2017 -- 2019}

IPBES Chapter Scientist of the chapter 4 of the IPBES Global Assessment

\subsection{2019 -- present}

Expert Member of the \textbf{IPBES Knowledge and Data Task Force}, nominated by Switzerland. 

\subsection{2021}

Member of Policy Network on Environment (PNE), Intersessional workstream of the Internet Governance Forum (IGF)

\section{Areas of Interest and expertise}

\subsection{Keywords}

open source tools; data and metadata management; ecological modelling; statistical computing; combined modelling and experimental approaches; invasive species management; spatial pattern analysis; decision support

\subsection{Research}


My current focus of work is in the field of data management, metadata management, workflows and data archival. Recent activities include development of a framework for continuous data analysis and archival which is used by an ongoing project (moved from development into usage and maintenance stage); development of data management guidelines for our working group (ongoing), development of a domain specific metadata scheme to make entering and useful metadata as easy as possible for the researcher while at the same providing all the information needed for a useful re-use of the archived data; management of the large scale literature review of the IPBES Global Assessment Chapter 4, including data collection, quality control, analysis, graphing and archival (finalising).

My previous research interest and focus is on 1) spatial modelling and analysis of pattern and processes and their integration with field experiments and observations, ranging from population (local) to ecosystem (regional) scale, 2) the impact of change (climate change, human impacts, alien spread, \ldots{}) and conscious human actions (management) on these pattern and ultimately on the function of these ecosystems and ecosystem services, and 3) the use of models in decision support of the management of natural resources.

During my research career I have developed and used different types of models, ranging from individual based models, over hybrid models using individual based approaches together with grid based elements, to pure grid based models. The systems studied ranged from populations and communities on the local scale to community dynamics (e.g. grassland - shrubland dynamics, two biocontrol species one invasive species system, spread of three alien invasive species) on the local scale and spread simulations of individual species on the national scale under different climate change scenarios. Most of my research included different management scenarios in the form of alien plant management actions.

All the simulation and analysis tools I use (and nearly all of the ones I used) are Open Source software (R, GRASS GIS, QGIS). This provides the flexibility to develop the simulation models and analysis protocols and pipelines without additional costs, distribute them freely and to enable others (scientists as well as other implementing agencies and usrs) to use and evaluate the code without limitations and without having to purchased specific software, i.e. reproducible research. Reproducible research includes for me to use scripts in analysis and generation of graphs and to make these as well as the simulation models available (as far as funder conditions allow this).

\subsection{IT}

During my research activities I obtained a broad expertise in programming in different languages, linking these through e.g. scripting to into automated process and data work flows, setting up and usage of virtual computing platforms (most recently S3IT Science Cloud) including docker, and virtual machines, setting up and maintenance of small MySQL and sqlite databases as well as SAMBA servers and trouble shooting of a wide variety of software as well as hardware problems. In addition, I maintain number of R packages for internal as well as public usage.  

Since I started in Zürich, a substantial part of my work involved research support, ranging from informal discussion on how workflows and scripts could be improved in regards to speed and data management / reproducibility aspects, designing and writing R packages to simplify or automate workflows up to designing workflows which use different computing platforms to the actual management of research data to finally submitting it to Zenodo for long term preservation.

During my career I was using different operating systems (in chronological order Windows, Linux, Mac) and obtained Expert knowledge of each. My daily OS is Mac and Linux. 

\subsection{Data Management}


In my role as an elected expert member of the Knowledge and Data Task Force of IPBES, I was leading the development of the first IPBES data management policy as well as, at the moment, leading the revision and continuing review of the policy to include aspects not included in traditional data management policies such as aspects specifically related to indigenous and local knowledge. 
In addition, I am involved in the development of educational material to facilitate the implementation of the policy, drafting of a data management vision for IPBES, and other activities concerning data management.

In my work in the Policy Network on Environment (PNE)  (Internet Governance Forum) I successfully promoted the FAIR use of data (Findable, Accessible, Interoperable and Reusable) as well as the CARE principles (Collective benefits, Authority of control, Responsibility and Ethics) to be included in the recommendations. Caused by the overall positive acceptance of the report, it was decided to continue the work in a Dynamic Coalition (https://www.intgovforum.org/en/content/dynamic-coalitions) which I am part in establishing it.


\section{Research Projects}

\begin{itemize}
\item Development of an R package for running a simulation reproducible as well as support for graphing of the results. 
\item Development and maintenance of an automatic data preparation and pre-processing pipeline to extract from regularly measured data and movies (every second day) research ready  as well as preservation ready data (open formats). The whole pipeline is implemented in a completely reproducible way (open code, docker, makefiles).
\item Development of a data management strategy for our research group and of tools to facilitate the provision and
improve the quality of metadata. \item Predictions in Chaotic systems
\item Management of the Literature Analysis (\textbf{IPPBES Global Assessment Chapter 4}) and the generated data and the Excel Database in a reproducible way for archiving as well as graphing of the results.
\item Analyse measured vertical wind profiles to improve the performance of a forest growth model (CASTANEA) in regards to energy balance
\item Modelling temporal and spatial dynamics of a range of different different alien species, alien control agents and 
management strategies
\item Optimising alien invasive plant management through modelling of temporal and spatial modelling
\item Modelling the role of seed dispersal in restoration and biological invasion
\end{itemize}


\section{Additional skills}

\subsection{Computer}

\textbf{Operating System} Expert Linux and Mac user; advanced Windows user

\textbf{Programming Languages} Extensive experience in programming in R,
Delphi / Pascal; basic usage of C and \LaTeX{}

\textbf{Programs} Extensive experience in R, GRASS, bash; Daily BBEdit user; Apple office programs; MS
Office programs; basic experience of QGIS and Arc-GIS

\subsection{Language}

\textbf{German} native language

\textbf{English} reading, writing and speaking fluent

\textbf{French} reading, writing and speaking fair

\section{Grants}

\subsection{2009 -- 2010}

NRF Freestanding Postdoctoral Fellowship

\subsection{1999 -- 2000}

Deutscher Akademischer Austauschdienst (DAAD: German Academic Exchange
Service) grant to conduct fieldwork for PhD at Gobabeb, Namibia.

\subsection{1996 -- 1997}

Deutscher Akademischer Austauschdienst (DAAD: German Academic Exchange
Service) grant to attend MSc in Conservation Biology course at UCT.

\section{Publications}

Data publication are not included. For an autogenerated list of the all publications see \url{https://orcid.org/0000-0002-7490-0066}

\titlespacing{\subsection}
            {0.4\textwidth}% max width of the title(for wrap/leftmargin shape)
            {5pt}% vertical space before the title
            {15pt}% separation between title and text
\subsection{Peer-reviewed Journals}

\begin{btSect}[elsarticle-harv]{MP_PeerReviewed}
\btPrintAll
\end{btSect}


\subsection{Book Chapters}

\begin{btSect}[elsarticle-harv]{MP_InBook}
\btPrintAll
\end{btSect}


\subsection{Conference proceedings}

\begin{btSect}[elsarticle-harv]{MP_Proceedings}
\btPrintAll
\end{btSect}


\subsection{Conference presentations}

Only first author, except invited keynote presentations
\begin{btSect}[elsarticle-harv]{MP_Presentations}
\btPrintAll
\end{btSect}

\subsection{Software Packages}

\begin{btSect}[elsarticle-harv]{MP_Software}
\btPrintAll
\end{btSect}


\subsection{Guest lectures}

\begin{btSect}[elsarticle-harv]{MP_GuestLecture}
\btPrintAll
\end{btSect}
\end{document}